%!TEX root = ../../PhD_thesis__Edouard_Leurent.tex

\makeatletter
\def\toclevel@chapter{-1}
\makeatother

\chapter{General Conclusion and Perspectives}
\label{chapter:conclusion}

\begin{flushright}
	\begin{tabular}{@{}l@{}}
		\emph{Nos équipiers}\\
		\emph{\hspace*{1.0cm}Sur les voies}\\
		\emph{\hspace*{0.5cm}Ralentissez}\\
	\end{tabular}

	Vinci Autoroutes\footnote{Writings collected by \href{https://twitter.com/pooredward/status/1273249408231124994}.
		{@pooredward}.}\hspace*{1cm}
\end{flushright}

\section{Conclusion on our contributions}
In this thesis, we proposed a learning-based approach to the problem of behavioural planning for autonomous vehicles, with a focus on situations where several drivers are interacting. Following an in-breadth (\Cref{chapter:2}) as well as in-depth (\Cref{chapter:3}) initial investigation, we identified a set of key issues which make this problem challenging. We now recall each of these subjects, and precise how we strived to address them in both the model-free methods presented in \Cref{part:2}, and the model-based methods of \Cref{part:3}.

\paragraph{Coupled social dynamics}
In dense traffic, the dynamics of distinct vehicles are locally coupled, due to how drivers react and adapt to their surroundings. Consequently, predicting the course or acting in a driving scene requires a \emph{social awareness} skill: the ability to accurately understand and exploit these couplings. %In particular, the ego-vehicle needs to assess the impact of its own decisions on the future behaviours of nearby drivers. 
In \Cref{chapter:4}, this skill was implemented through a \emph{social attention} mechanism in the policy architecture, which enables the agent to filter out irrelevant objects from a complex driving scene and retain only those that represent a risk of collision. In \Cref{chapter:5}, this coupling was made even more implicit by describing the motion of a vehicle $i$ through a dynamical model $\dot{x_i} = f_i(x)$ hat accepts the whole traffic state $x$ as an input.

\paragraph{Uncertainty due to human drivers}
Another key difficulty stems from the uncertainty of human behaviours. The traditional approach in \gls{RL} to account for this uncertainty is to incorporate stochasticity in the system dynamics, as we did in \Cref{chapter:5} where the objectives are formulated in terms of \emph{expected} rewards and costs. Incidentally, we also observed in \Cref{chapter:4} that our attention-based architecture is highly sensitive to ambiguous and disambiguated information, such as vehicles' destinations. In \Cref{chapter:7} however, we adopted another view and assumed that the dynamics were (close to) deterministic, but dependent on some unknown parameters --both continuous and discrete-- that could be estimated along the way.

\paragraph{Safety}
Three formalisations of safety:
1.	As a cost function, additional signal separate of rewards
2.	As a safe state-action space X, U
3.	As a worst-case outcome

\paragraph{Trade off between safety and efficiency}
Safety is always defined with respect to admissible uncertainty. The larger the set of scenarios we are willing to consider, the more conservative we need to be to ensure safety.
This trade-off is irreducible: consider the following example, you are driving…
Either you never allow yourself to overtake, or you lose safety. 
By pushing this reasoning, any kind of interaction is susceptible to lead to accidents, under unlikely events.
In that sense, \emph{absolute safety is not achievable}. To strike the right level of safety, you need to consider the right level of uncertainty, the right set of outcomes.
In Chapter 7, adjusting $\delta$ for continuous parameters (control the size of confidence set).
Adding / Removing (A, phi) hypotheses (e.g. a potential lane change of the front vehicle will prevent/allow the overtake).
If we are willing to take a model-free approach, the uncertainty comes from the environment, simulated or not. Then, the trade-off between safety and efficiency is controlled by adjusting beta. Like in countries… provides carmakers with…

\paragraph{Sample efficiency?}
Chap 4: An architecture tailored for the problem. invariant to permutations of the scene descriptions. Inductive bias: out of many vehicles perceived as inputs, only a few are relevant to the decision.

\section{Outstanding issues and perspectives}
Subjective, personal view on matters.
We identify several limitations.

\paragraph{From research to industrialisation}
Combining approaches: Complexity can be overwhelming. Every little aspect can be studied independently, but how to stuff together in one product?
Limitations of our methods. Model-free architectures: not very convincing, rate of collision still high.

\paragraph{Theoretical guarantees \vs empirical achievements}
Researchers strive to derive safety guarantees. 
But what are these guarantee worth when they rely on assumptions that are wrong. 
A HJI reachability analysis can do nothing against an unexpected event.
Still, principled ways to derive sound algorithms, whose properties can robustly generalize. Eg estimation and control in aerospace, where the dynamics is not really linear and the noise really gaussian.
Do you prefer a linear model with guarantees under false assumptions, or a Neural Network with no guarantee but a larger hypothesis class (reduced distance)?

\paragraph{Beyond simulation?}
How to use both ?
Simulation to develop a common sense of how the world works, but then transfer to real world data.
Algorithms specific to simulation: Pure Exploration Setting.
Offline RL. Safe Policy Improvement in the vicinity of known states.





