% --- French package
%\usepackage[french,greek]{babel}

% --- Front page Font
\newenvironment{frontfont}{\fontfamily{phv}\selectfont}{\par}
\DeclareTextFontCommand{\textFF}{\frontfont}

% --- Space between paragraphs/enums/items...
\setlength{\parskip}{0.5em}
\raggedbottom
\usepackage{etaremune}
\usepackage{enumitem}
\setlist[enumerate,itemize,description]{topsep=0em}

% --- Algorithms packages
\usepackage[algochapter,linesnumbered,commentsnumbered,inoutnumbered]{algorithm2e}

% --- Maths packages
\usepackage{amsmath}
% https://tex.stackexchange.com/a/12561/97964
\allowdisplaybreaks
\usepackage{amsthm}
\usepackage{amssymb}
\usepackage{bbm}
\usepackage{bm}
\usepackage{mathrsfs}
\usepackage{mathtools}
\usepackage{amsfonts}
\usepackage{empheq}
\usepackage{blkarray}

\usepackage{enumitem}
\usepackage{stmaryrd}

\usepackage{setspace}
\usepackage{stmaryrd}
\usepackage{cases}

% --- Figures and Captions
\usepackage{epstopdf}
\usepackage[small,bf,labelsep=endash,tableposition=bottom]{caption}
\usepackage{graphicx}
\DeclareGraphicsExtensions{.jpg,.pdf,.mps,.eps,.png}
% \usepackage{graphics}

% \usepackage{rotating}
% \usepackage{wrapfig}
\usepackage{float}
%\usepackage[caption=false,font=footnotesize]{subfig}
\usepackage[lofdepth,lotdepth]{subfig}
% \usepackage{subfigure}
% \usepackage{subcaption}
\newcommand{\subcaption}[1]{\caption{#1}}
\usepackage{cleveref}

% --- Figures Tikz
\usepackage{tikz}
\usetikzlibrary{positioning}
\usetikzlibrary{fit}
\usetikzlibrary{arrows}
\usetikzlibrary{shapes.geometric}
\usetikzlibrary{shapes.symbols}
\usetikzlibrary{backgrounds}
\usetikzlibrary{chains}
\usetikzlibrary{automata}
\usetikzlibrary{intersections}
\usetikzlibrary{decorations.pathreplacing}
\usepackage{tikzsymbols}  % http://texdoc.net/texmf-dist/doc/latex/comprehensive/symbols-a4.pdf for the \Coffeecup{} command!

% --- Color definition
\usepackage{xcolor}
\definecolor{Bleu}{RGB}{0,0,204}           % Define a new color rgb(0,0,204)
\definecolor{darkblue}{RGB}{0,0,126}       % Define a new color rgb(0,0,126)
\definecolor{Violet}{RGB}{102,0,204}       % Define a new color rgb(102,0,204)
\definecolor{deeppurple}{RGB}{102,0,204}   % Define a new color rgb(102,0,204)
\definecolor{darkgreen}{RGB}{0,100,0}      % Define a new color rgb(0,100,0)
\definecolor{lightgreen}{RGB}{185,220,87}      % Define a new color rgb(185,220,87)
\definecolor{gold}{RGB}{255,184,0}         % Define a new color rgb(255,184,0)
\definecolor{deepgold}{RGB}{205,105,0}     % Define a new color rgb(205,105,0)
\definecolor{Rouge}{RGB}{204,0,0}          % Define a new color rgb(204,0,0)
\definecolor{darkred}{RGB}{174,0,0}        % Define a new color rgb(174,0,0)
\definecolor{Highlight}{RGB}{251,0,0}      % Define a new color rgb(251,0,0)
\definecolor{gold}{RGB}{255,184,0}         % Define a new color rgb(255,184,0)

% --- Tables
\usepackage{booktabs}
\usepackage{multirow}
\usepackage{multicol}
\usepackage{rotating}
\usepackage{array}
\usepackage{multirow}  % https://en.wikibooks.org/wiki/LaTeX/Tables#Columns_spanning_multiple_rows
\usepackage{arydshln}
\usepackage{hhline}
\usepackage{makecell}
\usepackage[flushleft]{threeparttable}

% --- SI Units
\usepackage[binary-units,detect-all]{siunitx}

% --- Nomenclature
\usepackage{nomencl}
\makenomenclature

% --- Line Spacing
%\doublespacing
%\onehalfspacing
%\singlespacing

% --- Footnotes
%\usepackage[perpage]{footmisc}

% --- Bibliography
% Add `custombib' in the document class option to use this section
\ifuseCustomBib
	% \usepackage[square,sort,numbers,authoryear]{natbib}
	\usepackage[sort,numbers]{natbib}
	%\usepackage[backend=bibtex,style=alphabetic,natbib=true]{biblatex}
	%\renewbibmacro{in:}{}
	%\DefineBibliographyExtras{french}{\restorecommand\mkbibnamelast}
	%\bibliography{all-phd-thesis}
\fi

% changes the default name `Bibliography` -> `References'
\renewcommand{\bibname}{List of References}
\renewcommand{\contentsname}{Table of Contents}
\renewcommand{\nomname}{Abbreviations and Notations}

% --- User Defined
% \renewcommand{\partname}{Partie}
% \renewcommand{\chaptername}{Chapitre}
% \renewcommand{\listfigurename}{Liste des Figures}
% \renewcommand{\listtablename}{Liste des Tableaux}
% \renewcommand{\listlistingname}{Liste des Exemples de Programmes}
% \renewcommand{\appendixtocname}{Liste des Annexes}
% \renewcommand{\appendixname}{Annexe}

\newcommand*{\everymodeprime}{\ensuremath{\prime}}

% --- Table of Content
\setcounter{secnumdepth}{2}
\setcounter{tocdepth}{1}
% TODO FIXME change tocdepth to 1 for final version of the manuscript
\usepackage{minitoc}
\setcounter{minitocdepth}{1}

% --- Source code input, see https://en.wikibooks.org/wiki/LaTeX/Source_Code_Listings

\usepackage{framed}
%\usepackage[chapter,draft=false,final=true]{minted}

% https://tex.stackexchange.com/a/64845/97964
%\renewcommand{\listingscaption}{Code Example}% Listing -> Code Example
%\renewcommand{\listoflistingscaption}{List of Code Examples}% List of Listings -> List of Code Examples

% % https://tex.stackexchange.com/a/64845/97964
% \renewcommand{\lstlistingname}{Code example}% Listing -> Algorithm
% \renewcommand{\lstlistlistingname}{List of \lstlistingname s}% List of Listings -> List of Algorithms

% --- To Do notes
\ifsetDraft
	\usepackage[colorinlistoftodos]{todonotes}
	\newcommand{\TODO}[1]{\todo[inline]{TODO: #1}}
	\newcommand{\TODOE}[1]{\todo[inline,color=blue!40]{Edouard: #1}}
	\newcommand{\FIXME}[1]{\todo[inline]{FIXME: #1}}
\else
	\newcommand{\mynote}[1]{}
	\newcommand{\listoftodos}{}
	\newcommand{\TODO}[1]{}
	\newcommand{\TODOE}[1]{}
	\newcommand{\FIXME}[1]{}
\fi

%% Theorem English
%\newtheorem{theorem}{Théorème}[chapter]
%\newtheorem{lemma}{Lemme}[chapter]
% \newtheorem{proposition}{Proposition}[chapter]
\newtheorem{theorem}{Theorem}[chapter]
\newtheorem{assumption}[theorem]{Assumption}
\newtheorem{claim}[theorem]{Claim}
\newtheorem{corollary}[theorem]{Corollary}
\newtheorem{definition}[theorem]{Definition}
\newtheorem{defn}[theorem]{Definition}
\newtheorem{example}[theorem]{Example}
\newtheorem{lemma}[theorem]{Lemma}
\newtheorem{notation}[theorem]{Notation}
\newtheorem{proposition}[theorem]{Proposition}
\newtheorem{remark}[theorem]{Remark}

%\addto\captionsfrench{\def\figurename{{Figure}}}
%\addto\captionsfrench{\def\tablename{{Tableau}}}
%\addto\captionsfrench{\def\ALG@mname{{Tableau}}}
% \makeatletter
%\renewcommand{\ALG@name}{Algorithme}
% \makeatother
% \makeatletter
\newcommand{\clearevenpage}{%
	\clearpage%
	\if@twoside%
		\ifodd\c@page%
			\hbox{}\newpage%
			\if@twocolumn%
				\hbox{}\newpage%
			\fi%
		\fi%
	\fi%
}
\makeatother

\makeatletter
% https://tex.stackexchange.com/a/365249/97964
\def\cleardoublepage{%
	\clearpage%
	\if@twoside%
		\ifodd\c@page%
		\else%
			\hbox{}\thispagestyle{empty}\newpage%
			\if@twocolumn%
				\hbox{}\newpage%
			\fi%
		\fi%
	\fi%
}
% https://tex.stackexchange.com/a/365249/97964
\newcommand*{\cleartoleftpage}{%
	\clearpage%
	\if@twoside%
		\ifodd\c@page%
			\hbox{}\thispagestyle{empty}\newpage%
			\if@twocolumn%
				\hbox{}\newpage%
			\fi%
		\fi%
	\fi%
}
\makeatother

% https://tex.stackexchange.com/questions/25137/how-to-change-the-background-color-only-for-the-current-page
% \usepackage{afterpage}% for "\afterpage"
% \usepackage{pagecolor}% With option pagecolor={somecolor or none}
% \newcommand{\colorthispage}[1]{
% 	\newpagecolor{#1}%
% 	\afterpage{\restorepagecolor}%
% }
% \newcommand{\emptycolorpage}[1]{
% 	\newpagecolor{#1}%
% 	\newpage%
% 	\afterpage{\restorepagecolor}%
% }
% FIXME comment or uncomment if you need to compress the document
\newcommand{\emptycolorpage}[1]{
	% \null
	% \if@print
		% \clearpage%
		\cleardoublepage%
		% \cleartoleftpage%
		% \newpage%
		\thispagestyle{plain}%
		\null%
		% \pagecolor{#1}%  FIXME bring back color if I need
		\pagecolor{white}%
		% \null%
		\newpage%
		% \null%
		\pagecolor{white}%
	% \fi
}
% \newcommand{\emptycolorpage}[1]{\pagecolor{white}}

% Macros from Émilie Kaufmann's articles
\usepackage{0-Misc/macrosText}

% FIXME can I use the font I like?
% https://tex.stackexchange.com/questions/84770/using-palatino-and-euler-math#comment182397_84770
\usepackage{tgpagella}
% \usepackage{eulervm}


% pagecolorWhen  setting  the  background  color  with\pagecolor(a  commandfromcolor.sty), the first\pagecolormustprecede\usepackage{pdfpages}.\usepackage{color}\pagecolor{white}\usepackage{pdfpages}The color is nonrelevant, it can be changed afterwards by using\pagecoloragain.   Just  the  order  (first\pagecolorbefore\usepackage{pdfpages})  isimportant.
\pagecolor{white}
\usepackage{pdfpages}


% \usepackage{ifxetex,ifluatex}
% \ifnum 0\ifxetex 1\fi\ifluatex 1\fi=0% if pdftex
% 	\usepackage[T1]{fontenc}
% 	\usepackage[utf8]{inputenc}
% \else% if luatex or xelatex
% 	\ifxetex
% 		\usepackage{mathspec}
% 	\else
% 		\usepackage{fontspec}
% 	\fi
% 	\defaultfontfeatures{Ligatures=TeX,Scale=MatchLowercase}
% \fi
% % use upquote if available, for straight quotes in verbatim environments
% \IfFileExists{upquote.sty}{\usepackage{upquote}}{}
% % use microtype if available
% \IfFileExists{microtype.sty}{%
% \usepackage{microtype}
% \UseMicrotypeSet[protrusion]{basicmath} % disable protrusion for tt fonts
% }{}
