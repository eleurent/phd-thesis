% ************************** Thesis Acknowledgements **************************

\begin{acknowledgements}
Un concept central en apprentissage par renforcement est celui du \emph{``credit assignment''} («~attribution du mérite~»). Selon ce principe, lors de l'obtention d'une haute récompense, il convient de remonter l'historique des évènements survenus par le passé afin d'identifier ceux qui furent responsables de ce succès.
Prêtons-nous à l'exercice.

Tout d'abord, je dois beaucoup à mes parents, ainsi qu'à mon frère et mes sœurs. Leur affection et encouragements constants au fil des années m'ont permis de m'engager avec confiance dans cette aventure.

Mais je n'aurais pas entrepris cette thèse sans l'exemple éclatant des doctorants de Parrot, Gauthier Rousseau et Clément Pinard, ni le concours de Jill-Jênn Vie, Edouard Oyallon, Alberto Bietti et Michal Valko qui ont tous conspiré à me mener au laboratoire SequeL. Je remercie également l'examen du permis de conduire, auquel mes échecs répétés m'ont permis de développer un intérêt égoïste mais pragmatique pour ce sujet de thèse en particulier.

J'adresse maintenant mes plus hautes \emph{traces d'éligibilité} ainsi que mes plus chaleureux remerciements à mes encadrants. Odalric et Denis, j'admire profondément votre intégrité et votre rigueur scientifique, ainsi que l'étendue de vos connaissances, que vous savez mobiliser pour répondre à la moindre de mes interrogations avec une facilité déconcertante. Merci également pour votre ouverture d'esprit, dont témoigne particulièrement la rencontre de vos disciplines respectives et la confrontation fructueuse des points de vue et des méthodes qui en découle. Mais avant tout, je vous suis reconnaissant pour votre bienveillance, votre disponibilité et votre soutien appuyé lorsque j'en avais besoin.
Yann, je te remercie pour avoir monté ce projet ambitieux, et pour m'avoir accordé une pleine liberté dans mes recherches, bien qu'elles se soient parfois écartées des préoccupations très concrètes de l'ingénierie Renault. Enfin, Wilfrid je te remercie pour tes précieux conseils toujours pertinents.

I am also very grateful to Lucian Busoniu and Jorge Villagra for their thoughtful reporting on this manuscript, as well as to all members of the jury, Luce Brotcorne, Marc Deisenroth and Rosane Ushirobira, for their valuable feedback, and for giving their time and energy to make it possible for my defence to take place amid the turmoil of a global pandemic despite being under lockdown.

J'en viens à mes coreligionnaires de la pause-café, avec qui j'ai pu décompresser, partager mes intérêts, mes joies et mes peines. A SequeL tout d'abord, je remercie particulièrement Mathieu et Xuedong, camarades de la première heure avec qui j'ai entamé d'inoubliables marches aléatoires dans les montagnes de Stellenbosch~; Nicolas et Omar, avec qui j'ai eu le privilège et le plaisir de collaborer~; Guillaume et Lilian dont l'éloignement rendait la visite occasionnelle d'autant plus festive~; Nathan et Dorian, dont l'appétence pour le débat n'a d'égale que leur propension à le trancher à coups d'étude ad-hoc~;  Pierre Ménard dont la cinéphilie et l'hospitalité ont permis la renaissance en grande pompe du Cinéquel~; Pierre Schegg, compagnon d'armes avec qui nous défendons fièrement le bastion des roboticiens à SequeL, ainsi que Florian, Ronan, Merwan, Mahsa, Julien, Yannis, Reda, Romain, Jean, Sarah et Antoine. Sans oublier les chercheurs~: merci Emilie, Jill-Jênn et Philippe pour vos contributions à la vie du laboratoire et à sa convivialité, ainsi que pour les collaborations et conversations scientifiques.
À Renault maintenant, Jean, j'ai su dès notre rencontre à Munich que ton goût communicatif et intarissable pour la philosophie nous conduirait à de précieuses et mémorables conversations. Merci également à Lu, Clara, Edwin, Federico, Louis et Thomas pour nos fameux repas-mini-doc\textsuperscript{\tiny\textsf{TM}}. Je salue aussi les doctorants du CAOR, Philip, Marin et Florent, avec qui il est toujours agréable de discuter, aux Mines ou en conférence, de nos approches différentes à un sujet commun.

Mais il y a une vie en dehors du travail, et je dois ses meilleurs moments à mes amis, merci Luc, Pierre, Bertrand, Adrien, Benjamin, Mano, Quentin, Thomas D., Thomas L., Simon, Oriane, Ivain, Magali et Daniel.

Enfin, merci Ariane pour ton soutien indéfectible, pour toutes tes qualités que j'admire tant, et pour me donner envie d'être et de donner le meilleur de moi-même.
\end{acknowledgements}